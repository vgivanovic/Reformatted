%%% Time-stamp: <2023-10-07 09:30:24 vladimir>
%%% Copyright (C) 2019-2023 Vladimir G. Ivanović
%%% Author: Vladimir G. Ivanović <vladimir@acm.org>
%%% ORCID: https://orcid.org/0000-0002-7802-7970

\chapter{Discussion}\label{ch:discussion}

\begin{comment}
  With sheer repetition, and in the absence of evidence, a myth about K-12 education has taken hold: American public schools are abject failures. Something must be done to reign in the rapacious unions who protect and coddle incompetent teachers. Something must be done about lazy administrators who block progress. Something must be done to give back to parents control over their children's education. And that something is charter schools.

  Rocketship is one of the most successful charter school chains in the United States, but their success is not in educating elementary school children. Case in point: In August 2023, the Fort Worth Star-Telegram reported that only 23\% of Rocketship's students met state standards in reading and language arts \parencite{Allen.Ruiz2023} compared to 53\% statewide \parencite{TexasEducationAgency2023}.

  Instead, Rocketship's success is in making money.
\end{comment}

This dissertation's research question is ``Has Rocketship structured itself and its finances, to earn a return to investors, focusing especially on real estate transactions, and if so, how?'' In order to answer that question, my findings need to establish a convincing argument that
\begin{enumerate}
  \item Rocketship is profitable, and
  \item profitability is the most plausible explanation for how they've structured themselves and how they operate.
\end{enumerate}

The first criterion can be established by scrutinizing Rocketship's financial statements. Rocketship is profitable, and has been since it opened its first school: Its net assets have risen from just over \$2M in 2010 to nearly \$33.5M in 2022. Ideally, a document authored by Rocketship's founders where it is stated that Rocketship's purpose would be to make a profit would be sufficient to establish the second criterion. Unfortunately, no such document exists.

The nearest there is to a charter school agenda and rationale is the 300+ page report from GSV (Global Silicon Valley) Advisors \citetitle{Moe.etal2012} by \citeauthor{Moe.etal2012}. GSV Advisors are investment advisors to the \emph{digerati} of Silicon Valley, and their focus is making money. As an indication of just how much \citetitle{Moe.etal2012} focus on money, it is sufficient to note that the titles of eight out of the nineteen sections of that report are explicitly about markets and investments. Rocketship is one of the dozen or so ``education innovators'' given a thumbnail sketch. %\parencite[228]{Moe.etal2012}

Instead of searching for a ```smoking gun'' document which establishes the rationale for Rocketship's existence, we could perform a thought experiment. Suppose John Danner and Preston Smith came up with an idea for a charter school chain. What would they be thinking?
\begin{enumerate}
  \item Locate in high poverty areas.
  \begin{itemize}
    \item Parents are more likely to be desparate for a better education that their local, underfunded public schools can provide.
    NMTC requires investments in community which need economic help.
  \end{itemize}
  \item Enlist community members to (a) evangelize the charter and (b) provide a moral bulkwork against criticism from neighbors.
  \item Enlist a superlative propaganda machine (``The 74'', \url{the74million.org }) to promote ­ relentlessly ­ the value and virtues of charter schools
  \item Use standardized test scores (e.g. CAASPP) as the metric for success.
  \item Use, creatively, current charter school law to reduce financial risk to zero.
  \begin{itemize}
    \item Use state funding whenever possible: LCFF, SB740, CSFA, federal programs.
    \item Use conduit bonds to finance real estate acquisitions.
    \item Keep education and pedagogy separate from facilities.
  \end{itemize}
  \item Leverage the hatred that billionaires (e.g. the Waltons, Eli Broad, the Gates) have for any successful government program to obtain low or no cost grants or loans. (Successful government programs like public schools, Social Security, Medicare put paid to the notion that the market is the most efficient way of allocating economic resources.)
  \item Make performance promises that are never met.
  \item Overwhelm authorizers with incredibly length petitions.
\end{enumerate}

\section{How Rocketship Increases Profitability}\indent%
\label{sec:how-rock-incr}

\begin{itemize}
  \item Pay teachers less.
  \item Reduce human-mediated instructional time, and replace with technology. EVIDENCE.
  \item Use as many funding sources as possible.
  \begin{itemize}
    \item Accept outright grants to fund operations
    \item Issue bonds which will be forgiven, turning them into grants.
    \item Accept loans at low interest rates
    \item Issue conduit bonds to leverage the credit rating of state in which they operate.
    \item Take advantage of state rent subsidies.
      \end{itemize}
  \item Charge both managment fees (20\%) and administrative fees (15\%). 
  \item Take advantage of out-of-district SELPAs.
\end{itemize}

\section{Unavoidable Decreases in Profitablity}\indent%
\label{sec:unavoidable_decr_profit}

\begin{itemize}
  \item Charter schools must address issues of facilities and their attendent costs that typically are not issues for traditional public schools.
  \item Charter schools must borrow in the market for municipal bonds at interest rates that are higher than those of public school districts.
\end{itemize}

\section{Questions Left for Future Studies}\label{sec:future-studies}
\begin{itemize}
  \item Why did Rocketship open up schools in other states? Did they feel that they had tapped California as much as they could? Or was there another reason?
  \item What were the criteria Rocketship used to determine where to locate? Clearly California, and in particular near San José, was an obvious choice because the founders and funders lived in Silicon Valley, within communting distance of San José. Also, Rocketship would not be able to collect supplemental or concentration LCFF funding, nor would lenders be able to take advantage of the New Markets Tax Credit if a Rocketship school was not located in a economically depressed area, typically with a high percentage of minority students.
\end{itemize}

\begin{comment}
\subsection{Validity}
  \subsection{Reliability}
  \subsection{Limitations}
  \section{Evaluating the Results}
  \section{Rival Explanations}
  \section{Future Research}
\end{comment}



%%% Local Variables:
%%% mode: latex
%%% TeX-master: "Rocketship_Education-An_Exploratory_Public_Policy_Case_Study"
%%% End:
