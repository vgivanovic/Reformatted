% %%% Time-stamp: <2023-11-09 23:24:18 vladimir>
% %%% Copyright (C) 2019-2023 Vladimir G. Ivanović
% %%% Author: Vladimir G. Ivanović <vladimir@acm.org>
% %%% ORCID: https://orcid.org/0000-0002-7802-7970

\chapter{Discussion}%
\label{ch:discussion}%
\noindent\bigskip%

This chapter addresses this dissertation's research question:
\medskip%
\begin{quote}\OnehalfSpacing
  Has Rocketship structured itself to earn a return for its founders and investors, focusing especially on its real estate transactions?
\end{quote}
It looks more closely at the research question and discusses what kind of evidence would confirm or disconfirm the research question. Prior to looking at some approaches to answering the research question, it looks at three well-known approaches to making arguments for or against a proposal or policy. Then, the research question is answered and an argument is made to support that conclusion. Two final sections discuss what policy issues are raised during the examination of the data and the evidence, what further research is warranted. Lastly, this dissertation makes a brief conclusion.

\section{The Research Question}%
\label{sec:research-question}\indent%

The research question really is asking two questions:
\begin{enumerate}
  \item Has Rocketship structured itself to make money, and
  \item is real estate the vehicle that Rocketship uses to make money?
\end{enumerate}

\subsection{Types of Evidence}%
\label{sec:types-evidence}\indent%

\section{Summary and Key Findings}%
\label{sec:summary-key-findings}\indent%

\section{Approaches to Answering the Research Question}%
\label{sec:appr-answ-rese-quest}\indent%

\subsection{Rappaport's Rules}%
\label{sec:rappaports-rules}\indent%

Anatol Rapoport, a [mathematical] game theorist, proposed four rules that seek to increase understanding and avoid defensive responses. \citefirstlastauthor{Dennett2013} reformulated them in his book \citetitle{Dennett2013} \parencite{Dennett2013}:
\begin{enumerate}
  \item You should attempt to re-express your target’s position so clearly, vividly, and fairly that your target says, “Thanks, I wish I’d thought of putting it that way.”
  \item You should list any points of agreement (especially if they are not matters of general or widespread agreement).
  \item You should mention anything you have learned from your target.
  \item Only then are you permitted to say so much as a word of rebuttal or criticism.
\end{enumerate}
\medskip

Here, in rocketship's case, their argument might go along the following lines:
\begin{enumerate}
  \item Public schools in areas of poverty, for whatever reason, don't educate children well, and they educate children of color especially poorly. We (Rocketship) aim to change this by a combination of focus, targeted intervention, and technology. We are focused on doing whatever it takes to raise the educational attainment of our students, and we focus day in and day out on this one goal. Our pedagogy has two major aspects: We monitor and target with specialize intervention students who are not doing as well as we would like, and we use technology (computer-aided instruction) to tailor instruction to a specific child's needs. We believe that by controlling our facilities, we can remove a serious distraction that comes with sharing facilities with a public school district. Not only are we not beholden to the whims of the public school district, but we don't have to spend time preparing year after year a proposition 39 facilities request. We are never embroiled in petty disputes about interactions between public school students and our students because they never arise. Our results speak for themselves. All of our schools do better than their surrounding district and do better than the California average.
      \item I agree that some public school districts have failed in their principle duty to educate children. I also agree that staying the course is not an option because doing the same thing over and over and expecting different results is not likely to be successful, now or ever.
 
 I also agree that providing separate facilities for children rather than forcing them to share is more likely to be successful than otherwise. being able to control the teaching environment brings stability the classroom.
    \item Before writing this dissertation, I did not realize the difficulties in starting and operating a charter school. Funding is hard to come by; Facilities need to be constructed or modified well before classes begin; Children need to be enrolled; and parents need to participate. 
  \item All this being said, it is not at all clear that Rocketship's approach is a good one.
\end{enumerate}

\subsection{Toulmin Arguments}%
\label{sec:toulmin-arguments}\indent%

\subsection{Logic Models}%
\label{sec:logic-models}\indent%

\section{Answering the Research Question}%
\label{sec:answ-rese-quest}\indent%

\section{Public Policy Issues}%
\label{sec:publ-policy-chang}\indent%

\section{Areas for Future Research}%
\label{sec:issu-future-rese}\indent%

\section{Conclusion}%
\label{sec:conclusion}\indent%


%%% Local Variables:
%%% mode: latex
%%% TeX-master: "Rocketship_Education-An_Exploratory_Public_Policy_Case_Study"
%%% End:
