%%% Time-stamp: <2023-08-20 14:05:17 vladimir>
%%% Copyright (C) 2019-2023 Vladimir G. Ivanović
%%% Author: Vladimir G. Ivanović <vladimir@acm.org>
%%% ORCID: https://orcid.org/0000-0002-7802-7970

\chapter{Discussion}\label{ch:discussion}

With sheer repetion, and in the absense of evidence, a myth about K-12 education has taken hold: American puble schools are abject failures. Something must be done to reign in the rapacious unions who protect and coddle incompetent teachers. Something must be done about lazy administrators who block progress. Something must be done to give back to parents control over their children's education. And that something is charter schools.

Rocketship is one of the most successful charter school chains in the United States, but their success is not in educating elementary school children. Case in point: In August 2023, the Fort Worth Star-Telegram reported that only 23\% of Rocketship's students met state standards in reading and language arts \parencite{Allen.Ruiz2023} compared to 53\% statewide \parencite{TexasEducationAgency2023}. 

Instead, Rocketship's success is in making money. 

\section{Judging Case Studies}\label{sec:case-studies}

\begin{comment}
\subsection{Validity}
  \subsection{Reliability}
  \subsection{Limitations}
  \section{Evaluating the Results}
  \section{Rival Explanations}
  \section{Future Research}
\end{comment}

%%% Local Variables:
%%% mode: latex
%%% TeX-master: "Rocketship_Education-An_Exploratory_Public_Policy_Case_Study"
%%% End:
