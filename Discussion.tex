% %%% Time-stamp: <2024-02-02 21:21:01 vladimir>
% %%% Copyright (C) 2019-2024 Vladimir G. Ivanović
% %%% Author: Vladimir G. Ivanović <vladimir@acm.org>
% %%% ORCID: https://orcid.org/0000-0002-7802-7970

\chapter{Discussion}\label{ch:discussion}\indent%

\index{key findings|(}
This dissertation's epigraph is \textit{cui bono?}, who benefits? In particular, who benefits from Rocketship's structure 
and activities?

After a careful examination of Rocketship's finances, a key finding is that the analysis of Rocketship's \textit{publicly available} financial documents show that any profits derived from operations of the schools appear to have been obtained legally and there is no evidence that these profits have been distributed to private persons, at least in California. But since some portion of these profits have been used to open non-profit charter schools in other states, no finding can be made about the ultimate beneficiaries these funds without further investigation. Rocketship does have some questionable expenditures, namely \$7.5M spent on travel in 2019–2022. This is a large sum, and should also be investigated.

The absence of evidence of illegal activity by Rocketship is fortunate because California is notorious for charter school fraud. Nine years ago, \textcite{CPD2015} found \$81M in fraud, and in 2019, a single online charter school was discovered to have defrauded the State of California of \$400M. The report, \citetitle{CPD2015} estimated in 2015 that ``[t]he vast majority of this fraud perpetuated by charter officials will go undetected because California lacks the oversight necessary to identify the fraud'' \parencite[2]{CPD2015}. With tens billions of dollars of funding for charter schools in California alone coupled with lax oversight, the temptation for fraud must be great. 

The question remains: \textit{cui bono}? If no illegal activities were detected, who actually benefits from Rocketship's growth in net assets?\index{key findings|)}

\section{Arguments Useful for Answering the Research Question}\label{sec:appr-answ-rese-quest}\indent

The next two sections each present a different style answer to this dissertation's research question. The first follows rules defined by Anatol Rappaport, a [mathematical] game theorist, who sought to increase understanding and avoid defensive responses. \citeauthor{Dennett2013} reformulated them in his book \citetitle{Dennett2013} \parencite{Dennett2013}. These are referred to in the literature as Rappaport's Rules. The second style was created by Stephen Toulmin, a British philosopher interested in moral reasoning. He developed a method for making practical arguments where ethics and morality played a role. Arguments made in this style are called Toulmin arguments. %chktex 12

Note that neither Rappaport nor Toulmin claim that their argument styles are logically rigorous; they are just a way articulating a point of view in a cohesive manner which fosters understanding and an exchange of ideas rather than descending into shouting, \textit{ad hominem} attacks, or a shutdown of communication.

\subsection{Rappaport's Rules}\label{sec:rappaports-rules}\indent

\index{Rappaport-style arguments|(}%
Rappaport's rules for arguing are:
\begin{enumerate}[topsep=0.3\baselineskip,itemsep=0.25\baselineskip]
  \item Re-express your target’s position so clearly, vividly, and fairly that your target says, “Thanks, I wish I’d thought of putting it that way.”
  \item List any points of agreement (especially if they are not matters of general or widespread agreement).
  \item Mention anything you have learned from your target.
  \item Only then present your criticism.
\end{enumerate}
\medskip

Using the Rappaport model in [1–4] above, this dissertation argues that:
\begin{enumerate}
  \item Public schools, in areas of poverty, for whatever reason(s), fail to educate their students well, and they educate students of color especially poorly. Rocketship aims to change this by a combination of focus, targeted intervention, and technology. They are focused on doing whatever it takes to raise the educational attainment of their students, and they focus day in and day out on this one goal. Their pedagogy has two major aspects: They monitor and target with specialize intervention students who are not doing as well as they would like, and they use technology (computer-aided instruction) to tailor instruction to a specific child's needs at exactly the time they need intervention.\\ %chktex 36
  Rocketship believes that, by controlling their facilities, they can remove a serious distraction that comes with sharing facilities with a public school district. Not only are they not beholden to the whims of the public school district, but they do no have to spend time preparing year after year a Proposition 39 facilities request. They are never embroiled in petty disputes about interactions between public school students and their students because that simply never arises. Their results speak for themselves: All of their schools do better than their surrounding district and do better than the California average.

  \item It is true that some public school districts have failed in their primary duty to educate children. Staying the course, however, is not an option because doing the same thing over and over and expecting different results is not likely to be successful, now or ever. Rocketship provides an alternative for these children. Further, providing separate facilities for children as Rocketship does rather than forced sharing is more likely to be successful than otherwise. Being able to control the school environment brings stability the classroom.

  \item Starting and operating a charter school is not for the faint of heart. Funding for both operation and for facilities is hard to come by, and must be procured before a single student arrives on campus. Facilities need to be constructed or modified well before classes begin. Children need to be enrolled, and parents need to be persuaded to help out. Juggling priorities, contingencies, and expectations over and above those of established public schools is a full-time task, and it shows when the topics of board committee (executive, business, achievement, development) meetings are looked at.

  \item Even after re-expressing Rocketship's motivation and listing points of agreement and mentioning what was learned, there are some areas where criticism, some of it quite severe, is warranted.

  \begin{itemize}

    \item Even taking into consideration the acknowledged need for independent facilities, Rocketship spends an inordinate amount of time on topics that are not academic in focus. One can see this in the topic areas of its board committees, where only one of four has to do with academic achievement.
    \item Rocketship short-changes current students in order to create future students by allocating 20\% of revenue to facilities. Although this is legal, I suspect that the California Legislature did not expect one charter school to be begetting another, just as the Legislature didn't anticipate the damage that for-profit charter school chains or virtual charter schools would do.
    The percentage of revenue that Rocketship spends on administration (i.e.\ management), is unusually high at 14\% in 2022 \parencite[38]{RSEA2022}. General administrative expenses in public school districts are in the 5\% to 10\% range. So, immediately 35\% of revenue is siphoned off and does not go directly to educating children.

    \item Teachers at Rocketship schools in Santa Clara county have much less experience and are paid substantially less than, say, teachers in the San José Unified School District \parencite{SCCOE14-23}. If we believe that teachers are a key factor in delivering a quality education, then what kind of teachers Rocketship hires and how much it pays them are important determinants of the quality of the teaching staff and hence how well Rocketship educates its children. The evidence is not does not point in the direction of hiring high quality teachers. Rather teachers are hired because they are cheap.

    \item Rocketship has exaggerated how well its students do compared to public school students. The size of the discrepancy is 4× to 5× larger than what is reported in the literature. A possible explanation for this difference is that Rocketship is trying to convince people that they are both committed to ``eliminating the achievement gap in our lifetime'' and more importantly, successful at it. Left unexplored in this dissertation is how Rocketship justifies its assertion of excellence.\\
    According to the latest \citetitle{SCCOE2023}, comparing the percentage of Rocketship students who ``met/exceeded standards'' on the Smarter Balanced Summative Assessments (SBAC) test in English Language Arts (ELA) during the 2021–22 school year, only a single Rocketship school (out of eight Santa Clara County authorized schools) did better than public school students in Santa Clara County. In mathematics, the number of schools who did better was actually less than one; it was zero.\\
    Moving on to comparing how Rocketship schools did against state public schools, only one school exceeded the state average in ELA—by a single percentage point. Rocketship did better in mathematics: Five schools exceeded the state average in Mathematics. Granted, these results are not as bad as ACE Empower that managed only 19\% met or exceeded standards in ELA and 11\% in Mathematics, but for a chain of schools whose explicit goal is to close the achievement gap, Rocketship's scores are not encouraging \parencite{SCCOE2023}.\\
    Also discouraging is the trend in the last five years. In five schools, the percent of students who met/exceeded standards has fallen in ELA and all but one have fallen in mathematics. However, a deeper investigation into the scores is needed to determine if this because parents with children who were not doing as well as hoped for enrolled and thus brought the average scores down, or whether this is truly a reflection on the effectiveness of Rocketship's pedagogy.

    \item Rocketship's policy of owning its facilities has led to over \$185M\footnote{This figure is for all Rocketship schools, i.e.\ those in  California, Tennessee, Wisconsin, Washington, D.C. and Texas.} of debt, which comes to roughly \$32K per child. If that debt costs 3\% yearly to service, that is another \$5.4M per year that is not going directly toward educating children.
  \end{itemize}
\end{enumerate}\index{Rappaport-style arguments|)}

\subsection{A Toulmin-Style Argument}\label{sec:toulmin-arguments}\indent

\index{Toulmin-style arguments|(}
A Toulmin argument has six components, of which the first three are most commonly used. They are:

\begin{enumerate}
  \item Claims are statements or conclusions which must be justified.
  \item Grounds are the evidence (facts, data) that provide the basis for making the claim.
  \item Warrants are the connection between the claim and the evidence which backs up the claim.
  \item Backings buttress warrants.
  \item Rebuttals are counter-arguments, made in advance, to potential objections.
  \item Qualifiers express the degree of certainty about the claim.
\end{enumerate}

\prettyref{fig:toulmin-arg} below diagrams the relationship between the parts of a Toulmin argument.

\begin{figure}[htbp]
  \caption{\textit{The Toulmin Argument Schema}}\label{fig:toulmin-arg}%
  \centering%
  \copyrightbox[b]{\includegraphics[width=0.75\textwidth] {Toulmin Argument Schema}}{Adapted from ``Toulmin Argument'', Kalyca Schultz, Virginia Western Community College, CC-BY-SA in “Toulmin Argument Model” by Liza Long, Amy Minervini, and Joel Gladd is licensed under CC BY-NC-SA 4.0}
\end{figure}

\paragraph{Claim}\label{p:claim}
This dissertation's conclusion is that Rocketship is structured and operates to leverage various kinds of government funding (loans, grants, credit repair) to increase their real estate holdings which then allows them to open more charter schools.

\paragraph{Grounds}
Supporting that claim is the observation that Rocketship's main focus is not on academic success, but on acquiring real estate that is paid for by someone else. There are three such grounds which support the claim:
\begin{itemize}
  \item How they have structured themselves, and  how they have funded themselves.
  \item What California law, their Articles of Incorporation, and IRS Code say about individual enrichment.
  \item How they spend their time and where they spend their money.
\end{itemize}

Right from the beginning, Rocketship separated the financing, acquisition, and operation of their facilities from the running of a school; they borrowed heavily and made extensive use of government programs to fund their real estate projects. Rocketship has not availed itself (except in one instance) of alternative ways of acquiring facilities. They have not used Proposition 39 to obtain classrooms and fields from a school's home district. They have not tried to convert commercial office space into classrooms. They have not modified existing buildings to serve as schools or classrooms. Instead they bought land and built schools.

Since Rocketship Education and Launchpad Development, according to law and IRS regulations, as non-profit public benefit corporations, cannot distribute their income or assets according to law and to their Articles of Incorporation, they must have a different goal than enriching individuals. Rocketship Education's and Launchpad Development Company's Articles of Incorporation both contain language that is similar to, but more clearly expressed by ``Summit Public School's [Restated] Articles of Incorporation'':
\begin{quotation}
  Article III: The property of this Corporation is \textit{irrevocably} [emphasis added] dedicated to charitable purposes and no part of the \textit{net income or assets} [emphasis added] thereof or to the benefit of private persons except that the Corporation shall be authorized and empowered to pay reasonable compensation for services rendered and to make payments and distributions in furtherance of the purposes set forth in Article II of these Articles of Incorporation. \parencite[2]{SummitPublicSchools2017}
\end{quotation}
The Articles of Incorporation for both Rocketship and Launchpad are likely equivalent to the first paragraph of Article III of Summit's Articles of Incorporation quoted above, but those passages are not as explicit or definitive, and they are spread out over several articles and paragraphs.\footnote{Since both Rocketship and Summit make an exception to the prohibition of private gain for services rendered in their articles of incorporation, and entirely reasonable exception, and Launchpad's articles of incorporation do not, this raises the question of whether the \$300+23ary of Launchpad's CFO \parencite[7]{2019LDC990} in 2019–2020, for example, violates Launchpad's Articles of Incorporation.} California and IRS code forbid distribution of income or assets to anything but another public benefit corporation (see \prettyref{sec:real-estate-conv} below), individual enrichment is illegal. 


\paragraph{Backing}
Time and money are both in limited supply, so firms must decide how to allocate their time and money, and this allocation reveals the firm's true preferences and its goals.

\paragraph{Rebuttals}
A possible objection is that the three grounds given are neither necessary nor sufficient to prove the claim made on \pageref{p:claim}; there could be reasons other than expansionary reasons for Rocketship to have structured themselves they way they did. True, and also true, is that the way they spend their time and money could have been forced on them by circumstance or by law. Also true.

As a rebuttal to these objections, both overlook the element of choice that Rocketship had when organizing themselves. Rocketship knew what they were getting themselves into because their board of directors and founders had both real estate experience and charter school startup experience. They chose an organization that made RSED the decision maker and Launchpad the accumulator of value. They also understood that California law and IRS regulations didn't allow private enrichment because that was stated in their Articles of Incorporation. And lastly, they chose the purview of each board committee, only one of which concerned themselves with student achievement.

As an alternative, Rocketship could have chosen to make RSED (the parent organization) a 509(a)(3) supporting charity for individual 503(c)(3) non-profit public benefit corporations (the schools) and only provide requested services instead of forced management. Another alternative is that they could have even done away with RSED completely and just had a loose collection of independent charter schools. 

Rocketship had the freedom to choose their organizational structure. That combined with their knowledge that private enrichment is not allowed for non-profits, plus operating a sweep and having three times as many committees concerned with money and real estate than on student outcomes strongly suggests that Rocketship knew exactly what needed to be done, and then did it.
\index{Toulmin-style arguments|)}

\section{Answering the Research Question}%
\label{sec:answ-rese-quest}\indent%

As indicated earlier, this dissertation's research question can be parsed into three sub-questions:
\begin{enumerate}
  \item Has Rocketship structured itself to make a profit?\footnote{Public benefit corporations (non-profits) are allowed to make a profit, i.e.\ revenue exceeds expenses; they are, however, not allowed to \textit{distribute} that profit. Any profit must be use to further is public benefit purpose identified in their articles of incorporation.}
  \item If so, is real estate the vehicle that Rocketship uses to make money?
  \item Is this Rocketship's intent?
\end{enumerate}

The arguments given above indicate that the answers are: yes, yes, and yes.\index{research question!answer to}

Two questions remain: How and Why? It is not known how the owners and investors of Rocketship could convert the assets of Rocketship and Launchpad into transferable wealth. Perhaps they do not intend to. \prettyref{tab:types_conversion} below lists some of the forms and restrictions on converting real estate assets owned by a non-profit charter school.

\index{conversion, types of|(}
\noindent%
\begin{table}[ht]
  \caption[Types of Conversion]{\textit{Types of Conversion}}%
  \label{tab:types_conversion}
  \begin{tabulary}{\textwidth}{LLL}
    \toprule
    \textbf{Type of Conversion} & \mbox{\textbf{Allowed?}} & \textbf{Notes} \\
    \midrule
    Excessive salaries & No & Listed in Form 990. Monitored by the IRS \vspace{6pt} \\
    Sale of assets to a private entity or for-profit corporation & No & Prohibited by CA Government Code, IRS Code, Articles of Incorporation. Requires AG notification \vspace{6pt} \\
    Sale of assets to another non-profit & Yes & Provided the non-profit has similar public benefit objectives \vspace{6pt} \\
    Arms-length transactions & Yes & Allowed by conflict of interest laws \vspace{6pt} \\
    Conversion of property to condos or apts. & No & Non-profits restricted to charitable purposes \\
    \bottomrule
  \end{tabulary}
\end{table}
\index{conversion, types of|)}

\index{non-profit!private gain|(}
That then begs the question of why Rocketship is accumulating assets. According to California law and to IRS Code, charter schools are not allowed to transfer money to a for-profit entity or to private individuals. The option which makes the most sense in explaining Rocketship's structure and activities given the investments of Reed Hastings, Andre Agassi, the Walton Family Foundation, and others, all strong charter school supporters, is that Rocketship wants to become a self-perpetuating pipeline of new charter schools for the entire United States. Their expansion into Washington, D.C., Texas, Tennessee, and Wisconsin exactly fits this conjecture.

But it is also possible that the billionaires identified above have promised the founders or directors of Rocketship a bonus if they successfully create a model for self-perpetuating charter schools. Detecting such a payout, in another state or country (Panama, Cayman Islands) is hard and becomes harder the longer the time lag between leaving Rocketship and receiving the payout. I can imagine that tens of millions of dollars of bonus would not even rise to the level of rounding error in the books of billionaires.
\index{non-profit!private gain|)}

\section{Public Policy Issues}%
\label{sec:publ-policy-chang}\indent%

\index{public policy issues|(}
There are at least three major public policy issues that are raised by Rocketship's growth in assets and presumed expansion goals. First, are charter schools or charter school chains a net benefit to Californians? Secondly, is there too much opportunity for fraud, and lastly, should charter school chains in California use their assets (paid for by Californians) to create charter schools in other states? This last public policy issue could be more broadly construed to ask if any Californian tax dollars should leave the state with no benefit to Californians.

The first of these public policy issues, the net benefit of Californian charter schools to California, is beyond the scope of this dissertation because it would require a thorough analysis of not only the finances of charter schools, but also of the costs and benefits of the education they offer, their impact on public schools, and their impact on the communities they serve. The second and third issues—opportunities for fraud and creating charter schools in other states—are discussed in the sections below.
\index{public policy issues|(}

\subsection{Fraud}%
\label{sec:fraud}\indent%

\index{public policy issues!fraud|(}
The Association of Certified Fraud Examiners issues an annual \textit{A Report to the Nations} which is a global study on occupational fraud. Of interest is a diagram of the types of occupational fraud, reproduced below.
\begin{figure}[htbp]
  \caption{\textit{The Fraud Tree}}%
  \label{fig:fraud-tree}\centering%
  \copyrightbox[b] {\includegraphics[width=0.85\textwidth]{2022 Report to the Nations-p.10}}
  {Reproduced by permission with attribution: \textit{Occupational Fraud 2022: A Report to the Nations.} Copyright 2022 by the Association of Certified Fraud Examiners, Inc.}
\end{figure}
The reason the Fraud Tree is interesting is because public policy should make sure that laws or regulations are in place to prevent all types of fraud from occurring. The diagram shows three major categories, eight immediate subcategories, and 47 specific types of occupational fraud, fraud that occurs in the context of employment. Note that much of this fraud can be eliminated by mandating robust internal controls and thorough, independent audits, neither of which are completely in place for charter schools. This should be an active area for new or modified legislation.

Current California law and regulations are clearly insufficient to prevent massive fraud. The largest case so far involved the A3 Charter Schools in San Diego. The San Diego County District Attorney filed an 235 page indictment \parencite{SDDA2019} alleging a \$400M scheme to defraud the State of California. The two principle defendents pleaded guity \parencite{Taketa2021}.
Seventeen years earlier, the California State Auditor found that not only did authorizers and the California Department of Education fail to meet the student outcomes their charter required, but fiscal oversight was insufficient \parencite{CAStateAuditor2002}. In 2021, the Network for Public Education issued a report on how charter schools across the United States profit from lax oversight and regulations \parencite{Burris.Cimarusti2021}. Essentially, CMOs ``sweep'' all of the revenues of a charter school chain in return for administration, management, and marketing services. These CMOs may be for-profit corporations in other states. In 2016, Kamala Harris, then the California Attorney General, announced a settlement with K12 of \$168.5M because misleading advertising and misreported attendance numbers \parencite{Agpressoffice2016}.

One kind of fraud that is extremely difficult to monitor, as Justice Clarence Thomas has shown \parencite{Murphy.Mierjeski2023}, is that of gifts which do not involve the exchange of money, such as luxury vacations, private jet travel, and VIP passes to sporting events. These illegal gratuities leave no financial traces and it may be hard even to establish that a \textit{quid pro quo} exists. 

Some activities that are actually conflicts of interests can be masked so they appear as perfectly legal and normal. For example piloting EdTech software (Dreambox, Clever, Zeal Learning) is an activity that many public, private, and charter schools do, and this (in theory) improves the educational outcomes of students. But, if the founder of two EdTech companies (John Danner) was also the founder of Rocketship, there exists the possible of an appearance of a conflict of interest.\footnote{California conflict of interest law is complex and is beyond the scope of this dissertation \parencite{Chaney.etal2010}}.

With billions of dollars in funding every year in California alone, with accountability that is significantly less comprehensive than that of public schools, it is not surprising that fraud occurs in California's charter schools. Despite this, there is no indication that Rocketship or its principals engaged in fraudulent activities. However, as noted below, unaccounted for expenditures nonetheless leave open the possibility of private gain. 
\index{public policy issues!fraud|(}

\subsection{Real Estate Conversion}%
\label{sec:real-estate-conv}\indent%

\index{public policy issues!real estate conversion|(}
In the Public Interest, a national research and public policy organization, said:
\begin{quotation}\noindent
  Due to a loophole in [California] law, some private groups have used this public money to buy private property. While charter schools constructed with general obligation bonds cannot be sold or used for anything other than the authorized school, schools constructed with tax-exempt conduit bonds become the private property of the charter operator. Even if the charter is revoked, neither the state nor a local school district can take control of this property. Additionally, schools constructed with private funding subsidized by New Market Tax Credits or acquired with private funds but whose mortgage payments are reimbursed through the Charter Facilities Grant Program (known as “SB740”) are typically owned without restriction. In the event that such schools close down, their owners may be free to turn the buildings into condominiums or retail space, or sell them at a profit. In such cases, neither the school district nor any other public body is entitled to recoup the public dollars that have gone toward creating the facility. \parencite[6]{ITPI2018}
\end{quotation}

However, this is likely no longer correct because a plain reading of current law doesn't allow non-profits to benefit private individuals.\footnote{Here are a few excerpts from IRS regulations and California law.
  
  \begin{itemize}
    \item The IRS says this about 501(c)(3) organizations under the heading "Exemption Requirements":
    \begin{quote}\noindent
      To be tax-exempt under section 501(c)(3) of the Internal Revenue Code, an organization must be organized and operated exclusively for exempt purposes set forth in section 501(c)(3), and none of its earnings may inure to any private shareholder or individual.
      \ldots\\
      The organization must not be organized or operated for the benefit of private interests, and no part of a section 501(c)(3) organization's net earnings may inure to the benefit of any private shareholder or individual.
    \end{quote}
    \item The California Attorney General says in \textit{Attorney General's Guide for Charities} (2021):
    \begin{quote}
      Under California law, a public benefit corporation must be formed for public or charitable purposes and may not be organized for the private gain of any person. A public benefit corporation cannot distribute profits, gains, or dividends to any person. (p.3)
    \end{quote}
    and
    \begin{quote}
      Although public benefit corporations may qualify for important benefits, including exemption from income tax, they are subject to important legal restrictions. One critical restriction is that the assets of a public benefit corporation are considered irrevocably dedicated to charitable purposes, and cannot be distributed for private gain. (p.7)
    \end{quote}
    Finally, the AG says on p.14,
    \begin{quote}
      In addition, the founding document must require the organization to expressly dedicate its assets to exempt purposes in the event of dissolution.
    \end{quote}
    \item California Code, §§ 5130, 5237, 5410. Section 5410 says "No corporation shall make any distribution."
  \end{itemize}}
Consequently, income and assets, during the lifetime of the non-profit, or at dissolution, can only be transferred to other non-profits, and at dissolution, a 20-day prior notice must be given to the California Attorney General. The prohibition of private gain and the requirement to notify the CA Attorney General rule out Rocketship converting its properties to condominiums or retail space.

However, it is the case that lease payments under SB740 do continue even after any debt used to purchase real estate or to improve a property has been paid off. This is an income stream that provides no benefit to Rocketship's children because it merely increases the value of Rocketship itself without funding any academic programs.
\index{public policy issues!real estate conversion|)}

\section{Changes to Public Policy}%
\label{sec:chang-publ-policy}\indent

It is clear that public policy for the entire charter school sector needs to be updated with new or amended law and regulations to counter the innovation that people have shown in creating mechanisms to make money even when they should not. Only a few of these public policy changes are due to Rocketship and how it operates; the majority are due to other charter schools and charter school chains. If one believes that best disinfectant is sunshine, then the following changes which increase transparency should be considered:
\begin{itemize}
  \item Eliminate ``sweeps''
  \item Hold charter schools and charter school authorizers accountable
  \item Require a board to have at least one unaffiliated member
  \item Require auditors to express an opinion on the effectiveness internal controls.
\end{itemize}
These recommendations, elaborated on below, are similar to the policy recommendations of \textcite[44–46]{Baker.Miron2015}.\index{public policy!changes to}

\subsection{Eliminate Sweeps}%
\label{sec:eliminate-sweeps}\indent

\index{sweeps|(}
Sweeps are when all revenue is ``swept'' into a non-profit charter management organization (CMO) or a for-profit educational management organization (EMO). These management organizations are then responsible for all of the school's finances. Sweeps into EMOs are ripe for abuse because their operations are opaque, invisible to the public whose taxes fund charter schools. If one believes that publicly funded organizations should be answerable to their funders, taxpayers, then the finances of publicly funded organizations should be publicly visible, completely and comprehensively so. In addition, annual public audits of EMOs should be conducted by an independent auditor in exactly the same way that non-profit CMOs and public school are audited.
\index{sweeps|)}

\subsection{Hold Charter Schools To Their Charter}%
\label{sec:hold-charter-schools}\indent

\index{charter schools!accountability|(}
Charter schools exist because the state has granted them a charter to operate based on a petition submitted to an authorizer. The charter lays out the purpose and goals of the charter school and the ways that those goals will be met. In the last three years, the California Department of Education  recorded just 6 out of 57 closures (≈10\%) that were not voluntary. In 2022-23, there were ``more than 1,300 charter schools''\footnote{``Charter School Closures'' \url{https://www.cde.ca.gov/sp/ch/cefcharterschools.asp}} in California, so less than ½ of 1 percent of charter schools closed involuntarily. It beggars the imagination that 1294 charter schools were substantially meeting their charter obligations. What is much more likely is that authorizers were simply not holding the charter schools accountable.

As previously noted, in 2021-22, just a few Santa Clara County authorized Rocketship charter schools did better than the state average on the Smarter Balanced Assessment Consortium (SBAC) English Language Arts (ELA) tests that are part of the California Assessment of Student Performance and Progress (CASPP) and none did better than the average for all Santa Clara County schools. The results for Mathematics (Math) are better, but still not stellar.\footnote{Standardized tests reveal remarkably little about how well a child is doing in school. For example, totally absent from measurement are 4 of the 6 C's of 21\textsuperscript{st}  century learning \parencite{Hirsh-Pasek.etal2020}: collaboration, communication, creative innovation, and confidence. Giving standardized tests the benefit of the doubt, one might claim that content and critical thinking are measured. Of course the 6 C's are just one way of viewing a child's education. Many alternatives ways of measuring non-academic outcomes exist, none of which are used by standardized tests.}

If charter schools are doing poorly compared to public schools in their districts, then they are not fulfilling a key premise of why they were created in the first place: They were granted exemption from onerous [sic] laws and regulations in return for better performance than public schools. Laws and regulation should be changed to make clear what happens if a charter school performs poorly over a number of years, and authorizers should have clearly spelled out responsibilities. At stake is not only the education of children, but millions if not billions of dollars annually in subsidies, grants, and loans.
\index{charter schools!accountability|)}

\subsection{Unaffiliated Board Member}\indent%
\label{sec:unaff-board-memb}\indent%

\index{unaffiliated board member|(}
School boards consider both high level strategy and tactical minutia. They are responsible for every aspect of starting and running of a charter school, but currently there is no requirement that charter school boards contain an unaffiliated member. (Unaffiliated in this contexts means that they have no connection or relationship with the charter school, board members, staff, i.e.\ their relationship is at arms length. Parents of children who attend the charter school would not be unaffiliated under this definition.) If visibility and transparency is a goal, having an unaffiliated board member is necessary because board meetings, especially closed sessions, are where decisions are made and the rationale for those decisions are articulated. Thus the law should require at least one unaffiliated board member.
\index{unaffiliated board member|(}

\subsection{Effective Internal Controls}%
\label{sec:effect-intern-contr}\indent%

\index{charter schools!financial controls!effectiveness of|(}
It appears that routinely, independent auditors include a statement similar to the following:
\begin{quotation}\noindent
  In performing an audit in accordance with GAAS and Government Auditing Standards, we [\ldots]
  Obtain an understanding of internal control relevant to the audit in order to design audit procedures that are appropriate in the circumstances, but \textit{not for the purpose of expressing an opinion on the effectiveness of RSEA’s internal control}. Accordingly, no such opinion is expressed. [Emphasis added]
\end{quotation}
in every audit report. This is curious because auditors are uniquely positioned to express an opinion on the effectiveness of internal controls since they have seen many examples, both good and less good. If auditors do not evaluate the effectiveness of internal controls, who would be in a position to perform such an evaluation, fairly and comprehensively? The law should be extended to require auditors to express an opinion on the effectiveness of internal controls.
\index{charter schools|financial controls!effectiveness of|)}

\section{Areas for Future Research}%
\label{sec:areas-future-rese}\indent%

This dissertation has discovered the need for future research. Some areas are Rocketship-specific; others are applicable to charter schools in general.

\index{Rocketship!travel expenses|(}
Rocketship records \$7.5M spent on travel in the four years of 2019–2022 and \$2.6M in 2022 alone. These are significant sums, so a future investigation should ask for details to substantiate the propriety of those expenditures. Details like who traveled, what class did they travel in, where did the go and where did they come from, and was the travel justified. Since these travel expenditures appeared in an annual financial statement, the investigation might ask what object codes went into each of the categories listed in \prettyref{tab:consolidated_functional_expenses} on p.\pageref{tab:consolidated_functional_expenses}.
\index{Rocketship!travel expenses|)}

\index{charter schools!fiscal monitoring|(}
As far back as 2002, the California State Auditor found that
\begin{textquote}[\parencite{CAStateAuditor2002}][.]{Chartering entities lacked policies and procedures for fiscal monitoring and have not adequately monitored their charter schools}
\end{textquote}
In other words, internal and external controls are lacking.
\index{charter schools!fiscal monitoring|(}

\index{charter schools!disposal of assets|(}
A study of what happens to a charter's assets when a charter dissolves is needed because, although the law says that non-profits cannot benefit individuals, and non-profits must notify the Attorney General at dissolution, there is no effective monitoring nor is any enforcement mechanism. The result is that we really have no idea of what happened to the assets of the hundreds of charter schools which have closed since 1992.
\index{charter schools!disposal of assets|(}

\index{charter schools!net benefit of|(}
Another area for future research is calculating the net benefit of charter schools. Many studies have been made that examine the academic performance of charter schools vs public schools, and some studies have been made which quantify some of the costs to public schools when a charter school opens, but I am not aware of any study that incorporates \textit{all} of the costs and benefits of charter schools, even for a single school district, much less for an entire state or for all of the United States.
\index{charter schools!net benefit of|(}

\index{Rocketship!spreadsheet used for forecasting|(}
Yet another area that has not been adequately explored is the 26 sheet Excel spreadsheet that Rocketship used for forecasting and planning in 2009 (\url{https://docs.google.com/spreadsheets/d/1e5j8nn2Ofg6l5BlOaPi_qcByGH_OAt232RrvTkoJy2Q}). Some of its sheets forecast out to 2040, so it is clear that Rocketship is playing the long game. Understanding what Rocketship is thinking of doing would be revealed by a deep dive into this spreadsheet.

\section{Conclusion}%
\label{sec:conclusion}\indent%

The conclusion this dissertation reaches is that Rocketship's finances do not and are not designed to benefit private individuals, but they are intended to benefit other non-profit corporations, in particular, charter schools in states other than in California. If this is the case, then Californian taxpayers are effectively subsidizing charter schools in other states, a use of Californian tax dollars that I expect Californians would overwhelmingly reject.

Further, the differences in finances and especially accountability between public schools and charter schools admit the possibility for fraud in charter schools, a situation which is much less likely with the public school system because of their much greater transparency. It is difficult to prove that fraud has occurred, but it is even more difficult to prove that fraud has not occurred. Given the history of fraud in charter schools, and the amount of money which is sent to charter schools annually, and the lack of substantive financial controls and oversight, it would indeed be surprising if charter schools in California and Rocketship in particular were not engaged in some form or fraud.

%%% Local Variables:
%%% mode: latex
%%% TeX-master: "Rocketship_Education-An_Exploratory_Public_Policy_Case_Study"
%%% End:
