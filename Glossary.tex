%%% Time-stamp: <2023-09-22 10:48:39 vladimir>
%%% Copyright (C) 2019-2023 Vladimir G. Ivanović
%%% Author: Vladimir G. Ivanović <vladimir@acm.org>
%%% ORCID: https://orcid.org/0000-0002-7802-7970

\chapter{Glossary}\label{ch:glossary}

%%%
% \fxfatal{Italicize first use of glossary terms.}
% \fxfatal{Make description terms normal font.}
%%%

\begin{description}[nosep]\OnehalfSpacing%

% \medskip\item[term] Description...

\medskip\item[ADA] Average Daily Attendance, the method that the state of California uses to determine how many students are in a particular school. An alternative is to use the number of students enrolled, some of whom may attend sporadically but still need to be educated when they do attend.

\medskip\item[arm's length transaction] A transaction, usually financial, where all parties are independent and self-interested.

\medskip\item[basic aid] See ``community funded'', the preferred term.
  
\medskip\item[blended learning] A method of teaching where both in-person instruction and virtual instruction are used.

\medskip\item[bond] A bond is a loan whose terms (maturity date, interest rate) are fixed. Bonds are issued by a borrower (the debtor) to investors (the creditors) who are the source of the funds borrowed. The borrower is liable for repaying the debt,  usually on a fixed schedule. In return for getting the funds now, the borrower agrees to compensate the creditor by repaying both the amount loaned (the principal) and interest on the amount outstanding at an agreed upon (ther interest) rate.

For example, a school district (the borrower and debtor) might issue a bond that is bought by one or more investors (the creditors) and use those funds to build a school. The school district must then repay the bond, usually in equal monthly payments, that pay back the principal and any interest to the  purchasers of the bond.

\medskip\item[charter school] A quasi-private school that is publicly funded but privately run.

\medskip\item[chartering authority] A governmental entity that grants charter schools the authority to operate and which provides oversight. In California, a chartering authority could be a public school district, a county office of education, or the California Department of Education.

\medskip\item[charter management organization (CMO)] ``A non-profit organization that operates or manages a network of charter schools (either through a contract or as the charter holder) linked by centralized support, operations, and oversight \parencite{CDE2021b}''.

\medskip\item[charter school chain] One or more individual charter schools owned by or operated by a parent organization, i.e.  a charter management organization or a education management organization.

\medskip\item[community funded] In California, if the local property tax revenue of a public school district exceeds the state minimum educational guarantee under Prop. 98, that district is called ``community funded'' (formerly ``basic aid'').

\medskip\item[conduit bond] A conduit bond is a type of municipal bond where the bond is paid back, not by a public entity's reveue stream, but by a private entity, for example, a limited liability company or corporation. The public entity is merely the conduit, a passthrough entity, between investors and a private entity. (See \citetitle{GASB91-2019} for details on what qualifies as a conduit bond.) %chktex 8

the the source of the funds borrowed is not an investor, but merely a passthrough between the source of the funds and the borrower. For example, a state authority might buy a bond from a school district and pay the school district with taxpayer funds. The issuer of the bonds (the school district) then owes the state authority the bond principal plus interest.

\medskip\item[cream skimming] When charter schools select the best students to admit.

\medskip\item[cross-collateralization] A term from bond financing which indicates that an asset has been used as collateral in two different obligations.

\medskip\item[debt, convertible] An obligation (a loan or a bond) that might be converted into another form, in Rocketship's case, a grant or donation.

\medskip\item[debt, loans payable] An obligation (a loan or a bond) that must be repaid, usually with interest, within a certain period, often in equal monthly payments made over the term of the debt.

\medskip\item[double bottom line grantors] Grantors (philanthropies) which measure social impact in addition to fiscal performance.

\medskip\item[education management organization (EMO)] ``A for-profit entity that operates or manages a network of charter schools (either through a contract or as the charter holder) linked by centralized support, operations, and oversight.''\parencite{CDE2021b} %chktex 38

\medskip\item[general obligation bonds (GO)] General obligation bonds are tax-exempt bonds backed by a public entities revenues. California state law limits bond debt to 2.5\% of total assessed valuation for unified school district and 1.25\% for elementary and high school districts.

\medskip\item[municipal bond] A municipal bond is a bond issued by a public entity and bought by investors. The public entity (the debtor) borrows from investors (the creditor). Investors loan money to the public entity, and the public entity pays the investors back over time with interest. The public entity (usually) uses its revenue stream (i.e. taxes paid) to pay back the principal and interest.

\medskip\item[parcel tax] A property tax that is not based on the value of the property.

\medskip\item[philanthrocapitalism] Using a market capitalism approach in non-profits.

\medskip\item[portfolio school district] A collection of diverse charter schools managed as together.

\medskip\item[property tax] A tax based on the assessed value of a property.

\medskip\item[Proposition 13] Passed by California voters in 2000 as a constitutional amendment, Prop. 13 devastated funding to local governments, including school districts by limiting the property tax to 1\% of assess value and requiring a two-thirds majority to increase non-property taxes.

\medskip\item[Proposition 39] Passed by California voters in 2000 as a constitutional amendment and state statute, Prop. 39 mandates that public school districts \emph{must} provide reasonably equivalent facilities to charter schools if requested.

\medskip\item[Proposition 98] Passed by California voters 1988 as a constitutional amendment and state statute, Prop. 98 

\medskip\item[public school] Public schools are funded by taxpayers and are governed by a publicly elected Board of Trustees. Unlike charter schools, public schools accept any and all students who wish to enroll, at any time of year, regardless of race, national origin, sexual orientation, gender, religion, citizenship, ability, disability, or language proficiency. 

\medskip\item[related party transaction]  A transaction, usually financial, where all parties are not independent or are self-interested, i.e.  when the transaction is not an ``arm's length transaction''. A synonym for ``self-dealing''.

\medskip\item[revenue bonds] Tax-exempt bonds guaranteed by a schools revenue instead of by an LEA's property tax revenue.

\medskip\item[school choice] The umbrella term used by ``education reformers'' to put positive spin on the privatization of public education. Charter schools, school vouchers, and educational savings accounts are the most common forms of school choice.

\medskip\item[socio-economic status] A euphemism for wealth.

\medskip\item[student pushout] When charter schools push their lowest performing students out.

\medskip\item[tax-exempt conduit bonds] Bonds issued to make loans to entities other than state or local governments are known as
“conduit bonds” or “conduit issues” and state or local governments that issue these bonds are known as “conduit issuers.” Conduit issuers (usually) ensure that the revenues of the charter school are sufficient to pay off the conduit bond with interest.

\medskip\item[theory of action] A logical chain of reasoning that explains what needs to happen to go from a particular (current) social state to another (future) social state.

\medskip\item[trailer bills] Legislative bills which implement and fund elements of California's enacted budget.

\medskip\item[typical or neuro-typical children] Children without special needs.

\medskip\item[unduplicated pupils] The State of California augments school district revenue on a per pupil basis for every pupil that qualifies for free or reduced price lunch, or is an English language learner, or is a foster youth, but only an unduplicated basis. Notably, children with special needs are not considered \textit{unduplicated pupils}. Neither are homeless children.

\end{description}

%%% Local Variables:
%%% mode: latex
%%% TeX-master: "Rocketship_Education-An_Exploratory_Public_Policy_Case_Study"
%%% End:
